\chapter*{写在前面}
	\thispagestyle{fancy} 

	\lettrine{嗨}{,同学},你好.我是 Midnight Forever,一名和你一样,在题海中沉浮过的高考考生.这本书的诞生,源于在高考后,我突然奇想,能不能把我零散的笔记、错题和感悟整理成一本所谓的学霸笔记,或许还能赚点零花钱.然而,当我真正开始动手整理时,我翻阅了市面上许多资料,包括一数的“高考数学一本通”与我校老师们辛苦编撰的回归教材系列.我渐渐发现了许多优秀资料的闪光点,也看到了它们可以被改进、被超越的地方.
	
	于是我继续想,我否自己写本更系统、更优雅、更贴近备考真实需求的数学复习资料?于是这本书便产生了.相较于市面上琳琅满目的复习资料,这本书倾注了我对数学之美的理解和对学习效率的执着追求,并力求在以下几个方面做到极致:
	
	\begin{enumerate}
		\item \textbf{排版之美}
		我相信,一份赏心悦目的学习资料,能从心理上点燃各位的学习热情.本书采用\LaTeX{}排版,精心设计了字体(本书主要使用宋体,笔记和解析等部分使用楷体,粗体和重点使用黑体,统一分明)、版式与色彩,力求让每一个公式、每一幅图、每一段文字都清晰、优雅、和谐.
		
		\item \textbf{保姆级解析}
		受到一数教辅“蓝字”引导式解析的启发,我再此基础上,采取了一套更系统,更有条理的解析布局.有时我想你对着答案解析,每一步都看得懂,但合上书,自己却依然无从下手?这是因为许多答案只告诉你是什么,却没告诉你为什么这么想.本书的解析,也将遵循“\textbf{思路分析} $\to$ \textbf{核心技巧} $\to$ \textbf{解题步骤} $\to$ \textbf{归纳总结}”的模式,更重要的是,和你分享每一步背后的思考过程,让你真正建立起“我知道下一步要做什么”的解题逻辑流.本书的解析格式,是为了避免让你陷入“做到一半不知道自己要干嘛”,“不得不回头重新看去”的尴尬局面,减少乱试浪费的时间.
		
		\item \textbf{数形结合}
		笔者最讨厌的事之一就是解析答案说一大堆,结果一张图都没有. 不可否认这可能是解析在锻炼我们的画图能力,这本身也是一种学习的过程. 但是出于方便理解和本书的“优雅”哲学,我尽可能的配图了. 这一哲学将在解析几何与概率统计部分体现得淋漓尽致,在那里,我将使用图示帮助拆解概率统计的阅读理解题,并用图示将我们要介绍的解析几何知识与各种模型生动地体现出来.
		
		\item \textbf{聚焦基础,拒绝炫技.}
		高考数学,得基础者得天下.本书坚决摒弃各种偏难怪题和华而不实的“秒杀技巧”.我们只专注于那些占据了高考试卷绝大多数分值的核心基础知识、核心题型与核心思想方法.我们的目标是帮你夯实地基,让你在面对任何基础题和中档题时都能稳操胜券,为挑战更高难度的题目积蓄最强大的能量与自信.
	\end{enumerate}
	
	当然,作为一名仍在求学路上的学生,我的知识水平和视野终究有限,书中难免会出现疏忽与错漏.在此,我恳请各位同学和老师,在使用过程中若发现任何问题,能够不吝赐教,通过QQ(2454167821)与我联系.你们的每一条反馈,都将是这本书不断完善、走向完美的阶石.
	
	愿这本书,能成为你高三征途中一位温暖而有力的同伴.
	
	\vspace{2cm} 
	\begin{flushright}
		Midnight Forever (永远是深夜有多好) \\
		\today
	\end{flushright}
	
	
	\newpage
	\thispagestyle{empty}
	
	\begin{center}
		\vspace*{4cm} 
		
		\begin{tikzpicture}
			\node[font=\Huge\bfseries] at (0,0) {关于作者};
			\draw[thick, color=customcolor] (-3,-0.5) -- (3,-0.5);
		\end{tikzpicture}
		
		\vspace{2cm}
		
		\Large \textbf{Midnight Forever} \\
		\large (永远是深夜有多好)
		
		\vspace{1.5cm}
		
		\begin{minipage}{0.8\textwidth}
			\large
			2025届高考考生,来自某普通高中的一名文科生(政史地组合).
			
			\vspace{0.5cm}
			作为文科生,众所周知文科最能拉分的便是数学与英语两科,因此整个高中大部分时间我都投入于这两个科目,也取得了不错的成绩 .我也相信学习高中数学更是一种锻炼思维、欣赏理性的艺术的过程. 我今年在2025年高考新二卷中取得139分的成绩,在文科中我自认为也处于中上游水平了. 而且这段经历让我深刻体会到,扎实的基础和清晰的思维逻辑,远比盲目的题海战术更为重要.
			
			\vspace{0.5cm}
			编写此书,是希望将这份对数学的理解与热爱分享出去,帮助更多像我一样,并非天赋异禀,但愿意脚踏实地、探求甚解的同学. 希望能通过这份资料,让大家感受到数学的逻辑之美与解题的思维之乐.
			
			\vspace{0.5cm}
			感谢你的阅读,愿我们都能在数学的世界里,找到属于自己的那片星空.
		\end{minipage}
	\end{center}
	
	\newpage
	\thispagestyle{fancy} 
	
	\begin{center}
		\vspace*{4cm}
		
		\begin{tikzpicture}
			\node[font=\Huge\bfseries] at (0,0) {常见问题与解答};
			\draw[thick, color=customcolor] (-4.5,-0.5) -- (4.5,-0.5);
		\end{tikzpicture}
		
		\vspace{2cm}
	\end{center}

	{\large 
		
		\noindent 
		\textbf{Q: }完整的本书会包括哪些内容?
		\par\vspace{0.3cm} 
		\noindent
		\textbf{A: }将包括高中数学的所有内容.
		
		\par\vspace{1cm} 
		
		\noindent
		\textbf{Q: }本书是什么性质?参考书还是教科书?
		\par\vspace{0.3cm}
		\noindent
		\textbf{A: }笔者认为,参考书是你不需要从头读起,可以随时翻阅特定概念的工具书,它需要一定的知识储备;而教科书则逻辑紧密,需要从头学起,对预备知识要求不高.本书并非严格的其中一种,而是兼具两者之长.例如,在前面的章节中,你可能会看到“导数”的提前出现与应用,尽管它的系统讲解在后几章.高中数学各模块联系紧密,无法完全割裂.因此,我建议读者善于前后翻阅,体会作者在章节编排上的用心.
		
		\par\vspace{1cm}
		
		\noindent
		\textbf{Q: }本书是否免费?
		\par\vspace{0.3cm}
		\noindent
		\textbf{A: }并不.本书的制作,包括排版、编辑、绘图和校对等工作,均由我一人利用课余时间完成,投入了大量心血.如果您觉得本书对您有帮助,认可我的劳动与付出,可以花费几块钱购买完整版以支持我.当然,如果觉得编写得不尽如人意,市面上也有许多优秀的教辅可供选择,如一数的《必刷100讲》、新高掌等.
		
		\par\vspace{1cm}
		
		\noindent
		\textbf{Q: }为什么没在书里看见高考真题和模拟题?
		\par\vspace{0.3cm}
		\noindent
		\textbf{A: }因为还没加.得益于我在高三时所做的工作——系统地收集和分类了大量优质真题与模拟题,因此将其插入本书并不困难.真正的挑战在于,如何按照本书保姆级解析的风格,重写每一道题的解析.
		
		\par\vspace{1cm}
		
		\noindent
		\textbf{Q: }本书会包含“拉格朗日乘数法”、“奔驰定理”这类大招或二级结论吗?
		\par\vspace{0.3cm}
		\noindent
		\textbf{A: }会的,但会以星号(*)标记,表明其为拓展内容,并非人人必需.对于“奔驰定理”这类在特定地区(如浙江卷)较常出现,但在全国卷中几年下来几乎没考过的内容,本书也会收录,并加以注明.这些章节的加入,是出于知识的完整性和本书的“参考书”属性考虑.
		\par\medskip 
		关于“大招”和“二级结论”,尤其是在解析几何部分,我想特别说明一点.许多资料热衷于介绍各种技巧,但笔者认为,\textbf{对于绝大多数中等水平的同学而言,考场上有限的时间和巨大的压力,往往会让他们回归到最稳妥、最通用的方法,比如设点联立.(比如某位同学学了一堆技巧大招,最后在考场上还是设元联立...)}与其花费大量时间学习可能用不上、用不熟的技巧,不如将基础打得更牢.因此,本书在讲解解法时,会更侧重于两个层次:首先,清晰地讲授如何通过常规方法(如设元联立、求切线等)迅速拿到大部分基础分;其次,再探讨如何在此基础上进一步思考,向更高分迈进.我希望这种务实的编排思路,对读者有所帮助,且后来被证明是合理的.
		
	} 