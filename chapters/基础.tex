	\makechapteropener
	{一} 
	{基础} 
	{将一个难题分解为若干个更小的、可解决的部分,是攻克它的第一步。} 
	{勒内·笛卡尔 (René Descartes)} 
	
	\chapter{基础}
	
	\lettrine{在}{正式踏上}高中数学的复习征途之前,我们需要进行一次彻底的“热身”与“装备检查”.本章并非全新的知识,而是对那些贯穿整个高中数学、须臾不可离的基础技能与核心思想的再强化,基础不稳,寸步难行.让我们开始吧.
	
\begin{introduction}[知识概括]
	\item \textbf{因式分解:} \label{intro:factor} 系统掌握提公因式、公式法、十字相乘法、分组分解法四大基本技巧,并学会高次多项式的猜根与综合除法.
	\item \textbf{一元二次方程根的分布:} \label{intro:roots_dist} 这是连接函数、方程、不等式的核心难点.关键在于掌握判别式、对称轴、端点函数值这“三大抓手”,将根的几何位置问题转化为代数不等式组问题.
	\item \textbf{核心计算与变形技巧:} \label{intro:techniques} 熟练运用配方法、换元法、待定系数法等“万金油”式的代数技巧,它们是化繁为简、解决复杂问题的钥匙.
\end{introduction}
	
	\section{因式分解核心技巧}
	
	因式分解是多项式乘法的逆运算,是将一个多项式化为若干个整式的乘积的形式.它是解方程、化简分式、研究函数性质的基础.掌握其方法,是代数运算能力的基石.
	
	\subsection{提公因式法}
	这是所有因式分解方法中最优先考虑的方法.其核心是找到多项式各项中都含有的公共的因式(可以是单项式,也可以是多项式),然后将其提取出来.
	\begin{exercise}
		分解因式:$3x(a-b) - 2y(b-a)$.
	\end{exercise}
	\begin{solution}
		观察到 $(b-a) = -(a-b)$,所以两项中含有公因式 $(a-b)$.
		\begin{align*}
			3x(a-b) - 2y(b-a) &= 3x(a-b) - 2y[-(a-b)] \\
			&= 3x(a-b) + 2y(a-b) \\
			&= (a-b)(3x+2y)
		\end{align*}
	\end{solution}
	\qed
	
	\subsection{公式法}
	熟练运用乘法公式的逆运算,是快速分解特定形式多项式的关键.
	\begin{itemize}
		\item \textbf{平方差公式}: $a^2-b^2=(a+b)(a-b)$
		\item \textbf{完全平方公式}: $a^2 \pm 2ab + b^2 = (a \pm b)^2$
		\item \textbf{立方和/差公式}: $a^3 \pm b^3 = (a \pm b)(a^2 \mp ab + b^2)$
	\end{itemize}
	\begin{exercise}
		分解因式:$x^4 - 16$.
	\end{exercise}
	\begin{solution}
		该式可看作 $(x^2)^2 - 4^2$,首先应用平方差公式.
		\begin{align*}
			x^4 - 16 &= (x^2)^2 - 4^2 \\
			&= (x^2-4)(x^2+4) \\
			&= (x-2)(x+2)(x^2+4)
		\end{align*}
		注意:分解务必彻底,$x^2-4$ 还可以继续分解.$x^2+4$ 在实数范围内不能再分解.
	\end{solution}
	\qed
	
	\subsection{十字相乘法}
	这是分解二次三项式 $ax^2+bx+c$ 的核心技巧.其本质是寻找两个数,使其和为一次项系数,积为常数项与二次项系数的乘积.
	\begin{exercise}
		分解因式:$6x^2 - 5x - 4$.
	\end{exercise}
	\begin{solution}
		我们将首项 $6x^2$ 分解为 $2x \cdot 3x$,将常数项 $-4$ 分解为 $1 \cdot (-4)$ 或 $-1 \cdot 4$ 等.通过尝试,我们发现:
		\begin{center}
			\begin{tikzpicture}
				\node at (0,0.5) {$2x$};
				\node at (0,-0.5) {$3x$};
				\node at (2,0.5) {$1$};
				\node at (2,-0.5) {$-4$};
				\draw (0.3,0.5) -- (1.7,-0.5);
				\draw (0.3,-0.5) -- (1.7,0.5);
				\node[right, blue] at (2.5, 0) {交叉乘积和: $(2x)(-4) + (3x)(1) = -8x+3x = -5x$};
			\end{tikzpicture}
		\end{center}
		交叉乘积之和等于一次项系数 $-5x$,所以分解成功.
		$6x^2 - 5x - 4 = (2x+1)(3x-4)$.
	\end{solution}
	\qed
	
	\subsection{分组分解法}
	对于四项或四项以上的多项式,如果不能直接提公因式或套公式,可以尝试将其合理分组,使组内能分解,进而使组与组之间产生新的公因式.
	\begin{itemize}
		\item \textbf{二二分组}: 例如 $am+an+bm+bn = (am+an)+(bm+bn)$.
		\item \textbf{三一分组}: 通常是三项构成完全平方式,再与剩下的一项构成平方差.
	\end{itemize}
	\begin{exercise}
		分解因式:$x^2 - 4y^2 + 2x + 1$.
	\end{exercise}
	\begin{solution}
		观察到 $x^2, 2x, 1$ 这三项可以构成完全平方式.
		\begin{align*}
			x^2 - 4y^2 + 2x + 1 &= (x^2 + 2x + 1) - 4y^2 \\
			&= (x+1)^2 - (2y)^2 \\
			&= [(x+1)-2y][(x+1)+2y] \\
			&= (x-2y+1)(x+2y+1)
		\end{align*}
	\end{solution}
	\qed
	
	\subsection{高次多项式猜根与综合除法}
	对于整系数高次多项式,如果它有有理根,那么这个根一定是“常数项的因数 / 最高次项系数的因数”的形式(有理根定理).这为我们“猜根”提供了方向.
	\begin{definition}[综合除法]
		当猜出一个根 $x=a$ 后,说明多项式必有因式 $(x-a)$.综合除法是快速计算多项式除以 $(x-a)$ 的商式的竖式算法.
	\end{definition}
	\begin{exercise}
		分解因式:$x^3 - 2x^2 - 5x + 6$.
	\end{exercise}
	\begin{solution}
		常数项为6,其因数有 $\pm 1, \pm 2, \pm 3, \pm 6$.最高次项系数为1.
		尝试将 $x=1$ 代入:$1-2-5+6=0$.成功!所以必有因式 $(x-1)$.
		用 $(x-1)$ 去除原多项式:
		\begin{center}
			\begin{tabular}{c|cccc}
				1 & 1 & -2 & -5 & 6 \\
				&   & 1  & -1 & -6 \\
				\hline
				& 1 & -1 & -6 & \textbf{0} \\
			\end{tabular}
		\end{center}
		商式为 $x^2-x-6$.所以 $x^3 - 2x^2 - 5x + 6 = (x-1)(x^2-x-6)$.
		对 $x^2-x-6$ 使用十字相乘法,得 $(x-3)(x+2)$.
		所以,最终分解结果为 $(x-1)(x-3)(x+2)$.
	\end{solution}
	\qed
	
	\section{一元二次方程根的分布}
	
	这是一个综合性极强的问题,深刻体现了数形结合思想.它探讨的是一元二次方程 $ax^2+bx+c=0 \ (a>0)$ 的两个实数根 $x_1, x_2$ 相对于某个给定值 $k$ 的位置关系.
	
	\begin{note}[核心思想:几何问题代数化]
		解决此类问题的根本,在于将“根的分布”这一\textbf{几何位置描述},通过二次函数 $y=ax^2+bx+c$ 的图像特征,转化为关于系数的\textbf{代数不等式组}.我们必须学会用代数语言去描述几何形态.
	\end{note}
	
	\begin{theorem}[古希腊掌管根分布的神]
		要精确地“定位”抛物线的根,我们通常需要从三个方面进行约束:
		\begin{enumerate}
			\item \textbf{判别式 $\Delta=b^2-4ac$}:它掌管着“\textbf{有没有根}”.$\Delta \ge 0$ 是方程有实根的前提,是所有讨论的基石.
			\item \textbf{对称轴 $x = -\frac{b}{2a}$}:它掌管着“\textbf{根的整体位置}”.对称轴在 $k$ 的左边还是右边,决定了两个根是“整体靠左”还是“整体靠右”.
			\item \textbf{端点函数值 $f(k)$}:它掌管着“\textbf{根与特定点的相对位置}”.$f(k)$ 的符号决定了点 $k$ 是在两根之间 ($f(k)<0$),还是在两根之外 ($f(k)>0$).
		\end{enumerate}
	\end{theorem}
	
	\begin{figure}[H]
		\centering
		\begin{tikzpicture}[scale=1.2]
			\draw[->] (-3,0) -- (5,0) node[below] {$x$};
			\draw[->] (0,-2) -- (0,4) node[left] {$y$};
			% 抛物线
			\draw[blue, thick, smooth, domain=-0.5:4.5] plot (\x, {(\x-1)*(\x-4)});
			\node[blue, right] at (4.5, 2.25) {$y=f(x)$};
			% 根
			\fill (1,0) circle (1.5pt) node[below] {$x_1$};
			\fill (4,0) circle (1.5pt) node[below] {$x_2$};
			% 对称轴
			\draw[red, dashed] (2.5, -1.25) -- (2.5, 4) node[above, black, align=center] {对称轴 \\ $x=-\frac{b}{2a}$};
			% 端点值k1
			\draw[green!50!black, dashed] (0,0) -- (0,4);
			\node[below, green!50!black] at (0,0) {$k_1=0$};
			\node[left, green!50!black] at (0,4) {$f(k_1)>0$};
			% 端点值k2
			\draw[purple, dashed] (5,-1.5) -- (5,4);
			\node[below, purple] at (5,0) {$k_2$};
			\node[right, purple] at (5,-1.5) {$f(k_2)<0$};
		\end{tikzpicture}
	\end{figure}
	
	\begin{property}[常见根分布的等价条件 ($a>0$)]
		设方程 $ax^2+bx+c=0$ 的两根为 $x_1, x_2$.
		\begin{enumerate}
			\item \textbf{两根均大于 $k$} (两根同在k右侧):
			\begin{itemize}
				\item 逻辑分析:首先,得有根($\Delta \ge 0$).其次,两根整体在k右侧,意味着对称轴在k右侧.最后,为了防止一根在k左一根在k右,必须保证k点在两根之外,即$f(k)>0$.
				\item 条件组: $\begin{cases} \Delta \ge 0 \\ -\frac{b}{2a} > k \\ f(k) > 0 \end{cases}$
			\end{itemize}
			
			\item \textbf{一根大于 $k$,一根小于 $k$} (k在两根之间):
			\begin{itemize}
				\item 逻辑分析:只要保证k点在两根之间,抛物线开口向上,那么图像在k处的函数值必然小于0.而$f(k)<0$ 这一个条件,已经蕴含了图像必定与x轴有两个不同的交点(即$\Delta>0$).所以此条件最为简洁.
				\item 条件组: $f(k) < 0$
			\end{itemize}
			
			\item \textbf{两根均在区间 $(k_1, k_2)$ 内}:
			\begin{itemize}
				\item 逻辑分析:这是最全面的情况.首先要有根($\Delta \ge 0$).其次,根的整体位置要在区间内,即对称轴在 $(k_1,k_2)$ 之间.最后,要保证两根不跑到区间外面去,必须让区间的两个端点都在根之外,即 $f(k_1)>0$ 且 $f(k_2)>0$.
				\item 条件组: $\begin{cases} \Delta \ge 0 \\ k_1 < -\frac{b}{2a} < k_2 \\ f(k_1) > 0 \\ f(k_2) > 0 \end{cases}$
			\end{itemize}
		\end{enumerate}
	\end{property}
	
	\begin{exercise}
		若关于 $x$ 的方程 $x^2 - (m+1)x + m = 0$ 的两根均在区间 $(0, 2)$ 内,求实数 $m$ 的取值范围.
	\end{exercise}
	\begin{solution}
		设 $f(x) = x^2 - (m+1)x + m$.根据“两根均在区间 $(k_1, k_2)$ 内”的模型,令 $k_1=0, k_2=2$.
		我们需要同时满足以下四个条件:
		\begin{enumerate}
			\item \textbf{判别式}: $\Delta = [-(m+1)]^2 - 4m = m^2+2m+1-4m = (m-1)^2 \ge 0$.此条件对任意实数 $m$ 恒成立.
			\item \textbf{对称轴}: 对称轴为 $x = -\frac{-(m+1)}{2} = \frac{m+1}{2}$.
			要求 $0 < \frac{m+1}{2} < 2 \implies 0 < m+1 < 4 \implies -1 < m < 3$.
			\item \textbf{左端点值}: $f(0) = 0^2 - (m+1)\cdot 0 + m = m > 0$.
			\item \textbf{右端点值}: $f(2) = 2^2 - (m+1)\cdot 2 + m = 4 - 2m - 2 + m = 2-m > 0 \implies m < 2$.
		\end{enumerate}
		现在,我们将所有关于 $m$ 的条件取交集:
		$\begin{cases} m \in \mathbb{R} \\ -1 < m < 3 \\ m > 0 \\ m < 2 \end{cases}$
		在数轴上画出这些区间,易得它们的公共部分为 $(0, 2)$.
		因此,实数 $m$ 的取值范围是 $(0, 2)$.
	\end{solution}
	\qed
	
	\section{核心计算与变形技巧}
	
	\begin{definition}[配方法]
		配方法是代数恒等变形的核心,旨在创造出完全平方式,从而揭示二次函数的顶点、对称轴等几何性质,或是在解题中降次、简化.
		\textbf{核心步骤}:“提首(二次项系数)、折半(一次项系数)、再平方、加减平衡”.
		\begin{exercise}
			求函数 $y=2x^2-8x+5$ 的最小值.
		\end{exercise}
		\begin{solution}
			\begin{align*}
				y &= 2(x^2-4x) + 5 & \text{ (提首)} \\
				&= 2(x^2 - 4x + (-2)^2 - (-2)^2) + 5 & \text{ (折半-4得-2, 再平方)} \\
				&= 2[(x-2)^2 - 4] + 5 & \text{ (配成完全平方)} \\
				&= 2(x-2)^2 - 8 + 5 & \text{ (加减平衡)} \\
				&= 2(x-2)^2 - 3
			\end{align*}
			因为 $(x-2)^2 \ge 0$,所以 $y \ge -3$.当 $x=2$ 时,函数取得最小值-3.
		\end{solution}
		\qed
	\end{definition}
	
	\begin{definition}[换元法]
		当一个代数式结构复杂,但其中有某些部分重复出现时,可以用一个新变量(元)替代这个部分,从而简化问题,使其转化为我们熟悉的基本模型(如一元二次方程).
		\begin{note}[换元法的灵魂:换元必换范围]
			这是换元法的灵魂,也是最常见的陷阱.引入新变量后,必须立刻根据旧变量的范围,确定新变量的取值范围,否则极易产生增根.
		\end{note}
		\begin{exercise}
			解方程 $4^x - 3 \cdot 2^x + 2 = 0$.
		\end{exercise}
		\begin{solution}
			原方程可写为 $(2^x)^2 - 3 \cdot 2^x + 2 = 0$.
			令 $t=2^x$.因为指数函数 $y=2^x$ 的值域为 $(0, +\infty)$,所以我们必须附加条件 $t>0$.
			原方程化为关于 $t$ 的一元二次方程:$t^2 - 3t + 2 = 0$.
			解得 $(t-1)(t-2)=0$,即 $t=1$ 或 $t=2$.
			检验范围:$t=1$ 和 $t=2$ 均满足 $t>0$ 的条件,所以都有效.
			\begin{itemize}
				\item 当 $t=1$ 时,$2^x=1 \implies x=0$.
				\item 当 $t=2$ 时,$2^x=2 \implies x=1$.
			\end{itemize}
			所以原方程的解为 $x=0$ 或 $x=1$.
		\end{solution}
		\qed
	\end{definition}
	
	\begin{definition}[待定系数法]
		当已知一个函数或多项式的类型(如一次函数、二次函数、反比例函数),但其系数未知时,可以先设出其含有未知系数的标准形式,然后根据已知条件(如经过某些点、对称轴信息等)列出关于待定系数的方程组,解出系数即可.这是求函数解析式的最基本、最普适的方法.
	\end{definition}